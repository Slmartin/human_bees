\subsection{Alternative model}

In order to investigate the influence of external perturbations, a more robust model was developed as an alternative to the model proposed by XXXX. We name this model the MMM model.

\subsubsection{Basic principle}
The MMM model describes the dynamics of the work allocation of different taks in a society of workers. 
A worker is not equally skilled for all of the tasks; we define his \emph{productivity} for a task as the amount of work (or produced units) in a given time. 
The productivity of a worker for a given tasks evolves with time, depending on whether he is actually performing the task or not, which reproduces learning and forgetting. Similarly, the \emph{boredom} of the worker regarding the tasks evolves. For a worker, the boredom is equivalent to earning less money.
Each worker is remunerated for his work; as all the tasks are considered to be equally important for the society, they all deliver the same total amount of money, which is distributed to the workers proportionally to their productivity.
Every now and then, the worker is given the possibility to quit his task and choose another task, which he will do if he can get a better salary/boredom balance. 


\subsubsection{Implementation details}
The initial value for the productivity $P_{i,j}$ of worker $i$ at task $j$ is generated randomly according to the equation
\begin{equation}
	P_{i,j}(t_0) = 
\end{equation}
The initial productivity of a worker is directly related to his maximal productivity at this task by
\begin{equation}
	P_{i,j}^\textrm{max} = m \cdot P_{i,j}(t_0)
\end{equation}
