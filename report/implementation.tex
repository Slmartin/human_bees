%implementation
The implementation of the two models follows what has been described in detail in the previous sections. We will thus only briefly describe the main ideas of the implementation of the two presented models.

For the first model, there are three interdependent values which need to be solved at a time for each iteration step: $x_{ij}$, $\theta_{ij}$ and $s_{j}$. These variables represent vectors of length $N \cdot M$, $N \cdot M$ and $M$, respectively, and are calculated according to the differential equations given in Section 4.1. Therefore, we have  used Matlab‘s $\textit{reshape}$ function to combine all three variables into a single $2MN+M$ vector. This form of variable storage allows for parallel solving of the target variables, which was done according to the Euler-Maruyama method.

The PBM simulation is also based on an Euler scheme for the time evolution of the different quantities in the model. 
