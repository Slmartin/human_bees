In the last centuries, division of labor has become one of the most pronounced characteristics of our society. Since Prehistory, human society has become more and more complex, and the advance in technology has made it more difficult for a human being to master all tasks in the society. Today, this fact is illustrated by the astronomic number of existing jobs.

Division of labor is an important aspect of human society, but it has also been shown that even insect societies feature division of labor.%\cite{robinson}. 
It is therefore interesting to investigate the factors leading to this phenomenon, and to examine if a model would also be able to reproduce more complicated characteristics of a society such as social inequality. 

Theraulaz et al. have successfully developed a model describing the emergence of division of labor in insect societies.%\cite{theraulaz_response_1998}. 
Their model uses few parameters and describes time-dependent expressed through learning and forgetting. It is still quite simple and leaves room for extensions: for instance, only systems exhibiting two tasks have been investigated. The authors did account for the fact that skills are developped in time, but they used fix population sizes, ignored aging, considered an isolated system without external disturbances and did not account for personal preferences. 

In this work, we expand the model of Theraulaz et al., who investigated the division of labor in insect societies, based on variable response thresholds. We investigate the effect produced by perturbations on the system such as reinitialization of the abilities of a worker or increasing size of the society.
We also present the Productivity-Boredom-Money (PBM) Model, which we derived from the Theraulaz model in order to investigate other characteristics of a society. The PBM model was developed with the intention to describe a society of individualists driven by the search of the own happiness more than that of the society as is the case in the Theraulaz model. This for example allows the introduction of money in the model and makes the model able to give insights into more complicated properties of a system such as social inequality.
