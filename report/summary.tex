
The basic premises of the PBM model make it a model useful to investigate the individualist side of a society of workers. Whereas in the Theraulaz model, the workers work for the benefit of the society by working on different tasks simultaneously and doing what the society needs most (i.e.\ the tasks having large stimuli), the workers in the PBM model are incarnations of the \emph{homo economicus}, who tries to maximize the own benefit. Through its flexibility, the PBM model allows the investigation of a large range of aspects of a society of individualists and is able to simulate the origin of social inequalities. It is therefore rather oriented to the description of a human society than the Theraulaz model.
The PBM lacks man-made mechanisms such as social security benefits or altruism, but is able to give an insight into basic mechanisms of a society.
