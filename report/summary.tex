The extended Theralaux model was shown to be useful in a variety of circumstances. It is very important to understand that our discussions might have focused on bees, but can be extended to other colonies of insects and even higher organisms living in societies as well. We have shown that we are able to measure the welfare of a society and that we can model the different needs and urgencies towards specific goods and services achieved by tasks. The dynamics with a simple growth model could be implemented too. Additionally, we have tested the model with respect to a variety of perturbations. These tests allow for interpretations of external effects on a society, such as war, accidents and other incidents that remove workers from a given system.
We propose that it lies in the simplicity of our model that makes it applicable to narrowly defined and simple questions of societies for a wide range of different animals.

In order to better address the describtion of societies of higher organisms, especially humans, we also developed an alternative model, the PBM model. The basic premises of this model make it a model useful to investigate the individualist side of a society of workers. Whereas in the Theraulaz model, the workers work for the benefit of the society by working on different tasks simultaneously and doing what the society needs most (i.e.\ the tasks having large stimuli), the workers in the PBM model are incarnations of the \emph{homo economicus}, who tries to maximize the own benefit. Through its flexibility, the PBM model allows the investigation of a large range of aspects of a society of individualists and is able to simulate the origin of social inequalities. It is therefore rather oriented to the description of a human society than the Theraulaz model.
The PBM lacks man-made mechanisms such as social security benefits or altruism, but is able to give an insight into basic mechanisms of a society.




